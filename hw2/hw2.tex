
\documentclass[10pt]{article}

\usepackage{graphicx}
\usepackage{amsmath,amsfonts,amssymb}

\usepackage{hyperref}  % for urls and hyperlinks


\setlength{\textwidth}{6.2in}
\setlength{\oddsidemargin}{0.3in}
\setlength{\evensidemargin}{0in}
\setlength{\textheight}{8.9in}
\setlength{\voffset}{-1in}
\setlength{\headsep}{26pt}
\setlength{\parindent}{0pt}
\setlength{\parskip}{5pt}



% a few handy macros

\newcommand\matlab{{\sc matlab}}
\newcommand{\goto}{\rightarrow}
\newcommand{\bigo}{{\mathcal O}}
\newcommand{\half}{\frac{1}{2}}
%\newcommand\implies{\quad\Longrightarrow\quad}
\newcommand\reals{{{\rm l} \kern -.15em {\rm R} }}
\newcommand\complex{{\raisebox{.043ex}{\rule{0.07em}{1.56ex}} \hskip -.35em {\rm C}}}


% macros for matrices/vectors:

% matrix environment for vectors or matrices where elements are centered
\newenvironment{mat}{\left[\begin{array}{ccccccccccccccc}}{\end{array}\right]}
\newcommand\bcm{\begin{mat}}
\newcommand\ecm{\end{mat}}

% matrix environment for vectors or matrices where elements are right justifvied
\newenvironment{rmat}{\left[\begin{array}{rrrrrrrrrrrrr}}{\end{array}\right]}
\newcommand\brm{\begin{rmat}}
\newcommand\erm{\end{rmat}}

% for left brace and a set of choices
\newenvironment{choices}{\left\{ \begin{array}{ll}}{\end{array}\right.}
\newcommand\when{&\text{if~}}
\newcommand\otherwise{&\text{otherwise}}
% sample usage:
%  \delta_{ij} = \begin{choices} 1 \when i=j, \\ 0 \otherwise \end{choices}


% for labeling and referencing equations:
\newcommand{\eql}{\begin{equation}\label}
\newcommand{\eqn}[1]{(\ref{#1})}
% can then do
%  \eql{eqnlabel}
%  ...
%  \end{equation}
% and refer to it as equation \eqn{eqnlabel}.  


% some useful macros for finite difference methods:
\newcommand\unp{U^{n+1}}
\newcommand\unm{U^{n-1}}

% for chemical reactions:
\newcommand{\react}[1]{\stackrel{K_{#1}}{\rightarrow}}
\newcommand{\reactb}[2]{\stackrel{K_{#1}}{~\stackrel{\rightleftharpoons}
   {\scriptstyle K_{#2}}}~}

% Parts:

% set enumerate to give parts a, b, c, ...  rather than numbers 1, 2, 3...
\renewcommand{\theenumi}{\alph{enumi}}
\renewcommand{\labelenumi}{(\theenumi)}

% set second level enumerate to give parts i, ii, iii, iv, etc.
\renewcommand{\theenumii}{\roman{enumii}}
\renewcommand{\labelenumii}{(\theenumii)}




\begin{document}

% header:
\hfill \vbox{
\hbox{AMath 584 / Math 584}
\hbox{Homework \#2}
\hbox{Due 11:00pm PDT}
\hbox{Tuesday, October 25, 2016}
}


\vskip 0.5cm

{\bf Name:}   Yifei Guan

{\bf Netid:}  gyf135

\vskip 0.5cm

%--------------------------------------------------------------------------
\vskip 1cm
\hrule
{\bf Problem 1.}
Find (by hand) both the reduced and full SVD of the matrix
\[
A = \bcm 0&3\\ 2&0\\ 0&4\ecm.
\]
Remember that things should be ordered so that $\sigma_1 \geq \sigma_2$.

Use the SVD to determine the following:
\begin{itemize}
\item The best rank-1 approximation to $A$.
\item The 2-norm of $A$.
\item A basis for the linear subspace $\{x \in \reals^2:
      \|Ax\|_2 = \|A\|_2 \|x\|_2\}$ and the dimension of this space.
\item A basis for the orthogonal complement of the range of $A$.
\item A basis for the null space of $A^T$.
\end{itemize}

% uncomment the next two lines if you want to insert solution...
\vskip 1cm
{\bf Solution:}\\
Finding the SVD of A consists of calculating the eigenvalues and eigenvectors of $AA^T$ and $A^TA$. The eigenvectors of $A^TA$ make up the columns of V, the eigenvectors of $AA^T$ make up the columns of U and the singular values in $\Sigma$ are square roots of eigenvalues from $AA^T$ or $A^TA$. The singular values are diagonal entries of the $\Sigma$ matrix and are arranged in descending order. The singular values are always real numbers. Since the matrix A is a real matrix, U and V are also real.\\
\centerline {Let $B = AA^T = \bcm 0&3\\ 2&0\\ 0&4\ecm \bcm 0&2&0\\ 3&0&4\ecm = \bcm 9&0&12\\0&4&0\\12&0&16\\\ecm$}\\
\centerline {Let $C = A^TA = \bcm4&0\\0&25\ecm$}\\
To find the eigenvalues and eigenvectors, we can solve $det(B - \lambda I) = 0$ for $\lambda$ where $det$ denotes the determinant and $\lambda$ is the eigenvalues of matrix B. Then we find that the eigenvalues of B are $\lambda_1 = 25, \lambda_2 = 4, and \lambda_3 = 0$. Note that $\lambda_1 $ and $\lambda_2$ are the eigenvalues of matrix C. For eigenvalues, we can calculate the eigenvectors respectively. As for $\lambda_1$, $BV_{B1} = \lambda_1V_{B1} \Rightarrow V_{B1} = \bcm 0.6\\0\\0.8\ecm$ and $CV_{C1} = \lambda_1V_{C1} \Rightarrow V_{C1} = \bcm 0\\1\ecm$. As for $\lambda_2$, $BV_{B2} = \lambda_2V_{B2} \Rightarrow V_{B2} = \bcm 0\\1\\0\ecm$ and $CV_{C2} = \lambda_2V_{C2} \Rightarrow V_{C2} = \bcm 1\\0\ecm$. As for $\lambda_3$, $BV_{B3} = \lambda_3V_{B3} = 0 \Rightarrow V_{B3} = \bcm 0.8\\0\\-0.6\ecm$.\\
Now that we have calculated the eigenvalues and eigenvectors. The full SVD of matrix A can be written as:\\
\centerline {$U = \bcm 0.6&0&0.8\\0&1&0\\0.8&0&-0.6\ecm$, $\Sigma = \bcm 5&0\\0&2\\0&0\ecm$ and $V = \bcm 0&1\\1&0\ecm$}\\
Double check that $U\Sigma V^* = A$ (Always needed since the eigenvectors can be of opposite sign).\\
Since the matrix A is of rank 2, there are only two singular values. The reduce SVD of matrix A can be interpreted as:\\
\centerline {$\hat{U} = \bcm 0.6&0\\0&1\\0.8&0\ecm$, $\hat{\Sigma} = \bcm 5&0\\0&2\ecm$ and $V = \bcm 0&1\\1&0\ecm$}\\
Double check that $\hat{U}\hat{\Sigma} V^* = A$\\

The best rank-1 approximation is thus $u_1\sigma_1 v_1^* = 5 \bcm 0.6\\0\\0.8\ecm \bcm 0&1\ecm = \bcm 0&3\\0&0\\0&4\ecm$\\
where $u_1$ and $v_1$ is the first column in matrix $U$ and $V$ respectively. And $\sigma_1$ is the largest singular value\\

The 2-norm of A: $\parallel A\parallel_2 = \sigma_1 = 5$\\

To interpret the subspace, let $x = \bcm x_1\\x_2\ecm$, and $Ax = \bcm 3x_2\\2x_1\\4x_2\ecm$. Then the subspace should satisfy $[(3x_2)^2+(2x_1)^2+(4x_2)^2]^\frac{1}{2} = 5 \times (x_1^2+x_2^2)^\frac{1}{2} \Rightarrow x_1 = 0$ and $x_2 \in R$.\\
Therefore, the basis of the subspace is $\{\bcm 0\\1\ecm\}$ and thus the subspace is just a line. Therefore the dimension of subspace is 1.\\

The $range(A) = Ax = \bcm 3x_2\\2x_1\\4x_2\ecm$, $\forall x_1,x_2 \in R$. And thus the orthogonal complement is $\bcm y_1\\y_2\\y_3\ecm$, $\{y_1, y_2, y_3 \in \mathbb{R} : x^Ty = 0\}$. In order to make the equation $x^Ty = 0$ true $\forall x_1,x_2 \in \mathbb{R}$, $y_2 = 0$ and $y_3 = -\frac{3}{4}y_1$. Therefore, the orthogonal complement of the range of matrix A is $Y = y_1\bcm 1\\0\\-\frac{3}{4}\ecm$. And the basis of it is $\bcm 1\\0\\-\frac{3}{4}\ecm$\\

$A^T = \bcm 0&2&0\\3&0&4\ecm$ and let $A^Tx = 0 \Rightarrow \bcm 0&2&0\\3&0&4\ecm \bcm x_1\\x_2\\x_3\ecm = \bcm 2x_2\\3x_1+4x_3\ecm = \bcm 0\\0\ecm$. Solve the equation, we have $x_2 = 0$ and $x_3 = -\frac{3}{4}x_1$. Therefore, the basis for the null space of $A^T$ is $\bcm 1\\0\\-\frac{3}{4}\ecm$\\

In fact, it can be shown that if the subspace is described as the range of a matrix, then the orthogonal complement is the set of vectors orthogonal to the rows of $A$, which is the nullspace of $A^T$




% insert your solution here!


%--------------------------------------------------------------------------
\vskip 1cm
\hrule
{\bf Problem 2.}
Determine by hand the SVD of the matrix
\[
A = \bcm 17&1\\ 6&18\ecm.
\]
Hint: one right singular vector is
\[
v_1 = \bcm 1/\sqrt{2}\\ 1/\sqrt{2} \ecm.
\]


% uncomment the next two lines if you want to insert solution...
\vskip 1cm
{\bf Solution:}
Based on the fact that the right singular vector is unitary, we can solve it with $v_1 = \bcm \frac{1}{\sqrt{2}}\\\frac{1}{\sqrt{2}}\ecm$\\, which gives:\\
\centerline {$V = \bcm \frac{1}{\sqrt{2}}&\frac{1}{\sqrt{2}}\\\frac{1}{\sqrt{2}}&-\frac{1}{\sqrt{2}}\ecm$} \\
\centerline {$AV = \bcm 17&1\\6&18\ecm \bcm\frac{1}{\sqrt{2}}&\frac{1}{\sqrt{2}}\\\frac{1}{\sqrt{2}}&-\frac{1}{\sqrt{2}}\ecm = \bcm\frac{18}{\sqrt{2}}&\frac{16}{\sqrt{2}}\\\frac{24}{\sqrt{2}}&-\frac{12}{\sqrt{2}}\ecm $}\\

Notice that $U\Sigma = AV = \bcm\frac{18}{\sqrt{2}}&\frac{16}{\sqrt{2}}\\\frac{24}{\sqrt{2}}&-\frac{12}{\sqrt{2}}\ecm$. Since $U$ is also a unitary matrix, it can be solved that $U = \bcm \frac{3}{5}&\frac{4}{5}\\\frac{4}{5}&-\frac{3}{5}\ecm$ and $\Sigma = \bcm \frac{30}{\sqrt{2}}&0\\0&\frac{20}{\sqrt{2}}\ecm$
% insert your solution here!


%--------------------------------------------------------------------------
\vskip 1cm
\hrule
{\bf Problem 3.}
Exercise 6.1 in Trefethen and Bau.  In addition,
illustrate with a sketch showing the effect of $A = I-2P$ on a typical vector
$v$ for the case
\[
P = \frac 1 5 \bcm 1&2\\2&4\ecm.
\]
Also explain why a matrix $A$ of this form is sometimes called a
``reflector''.  {\bf Hint:} See also Figure 10.2 in the book and the
discussion for the case of a ``Householder reflector''.


% uncomment the next two lines if you want to insert solution...
\vskip 1cm
Exercise 6.1:\\
If P is an orthogonal projector, then I - 2P is unitary.\\
{\bf Solution:}\\
\centerline {$(I-2P)(I-2P)^* = (I-2P)(I-2P^*) = I - 2P - 2P^* + 4PP^*$}\\
And since for orthogonal projector, $P = P^* = P^2$, the equation above becomes:\\
\centerline {$(I-2P)(I-2P)^* = I - 2P - 2P + 4P^2 = I - 4P + 4P = I$}\\
Therefore, $I-2P$ is unitary.\\

Let $P = \frac{1}{5}\bcm 1&2\\2&4\ecm$ and then $A = I - 2P = \bcm 1&0\\0&1\ecm - \frac{2}{5}\bcm 1&2\\2&4\ecm = \bcm \frac{3}{5}&-\frac{2}{5}\\-\frac{2}{5}&-\frac{3}{5}\ecm$\\
As for a general vector $v = \bcm v_1\\v_2\ecm$, we have $Av = \bcm \frac{3}{5}v_1 - \frac{2}{5}v_2\\- \frac{2}{5}v_1 - \frac{3}{5}v_2\ecm$


% insert your solution here!

%--------------------------------------------------------------------------
\vskip 1cm
\hrule
{\bf Problem 4.}
Exercise 6.3 in Trefethen and Bau.  Use the SVD to do this.


% uncomment the next two lines if you want to insert solution...
\vskip 1cm
Exercise 6.3\\
Given $A \in \mathbb{C}^{m \times n}$ with $m leq n$, show that $A^*A$ is nonsingular if and only if $A$ has full rank.\\
{\bf Solution:}\\
Perform SVD for matrix $A$ and let $A = U\Sigma V^*$. Then $A$ is of full rank if and only if $\Sigma$ is of full rank. Since $A^*A = V\Sigma^*\Sigma V^*$, to regard this as an eigenvalue decomposition, $A^*A$ is non-singular if and only if all eigenvalues are non-zero which means the diagonal matrix $\Sigma$ doesn't have zero on its diagonal and thus it is full rank. Therefore, $A^*A$ is nonsingular if and only if $A$ has full rank.\\
% insert your solution here!


%--------------------------------------------------------------------------
\vskip 1cm
\hrule
{\bf Problem 5.}
Exercise 6.4 in Trefethen and Bau.


% uncomment the next two lines if you want to insert solution...
\vskip 1cm
Exercise 6.4\\
$A = \bcm 1&0\\0&1\\1&0\ecm$ and $B = \bcm 1&2\\0&1\\1&0\ecm$
{\bf Solution:}\\
$range(A) = <\bcm \frac{1}{\sqrt{2}}\\0\\\frac{1}{\sqrt{2}}\ecm,\bcm0\\1\\0\ecm>$\\
and $range(B) = <\bcm \frac{1}{\sqrt{2}}\\0\\\frac{1}{\sqrt{2}}\ecm,\bcm\frac{2}{\sqrt{5}}\\\frac{1}{\sqrt{5}}\\0\ecm>$\\
(a)From equation 6.13, \\
\centerline {$P_A = A(A^*A)^{-1}A^* = \bcm 0.5&0&0.5\\0&1&0\\0.5&0&0.5\ecm$}\\
And the image under $P$ of the vector $(1,2,3)^*$ is:\\
\centerline {$ \bcm 0.5&0&0.5\\0&1&0\\0.5&0&0.5\ecm \bcm 1\\2\\3\ecm = \bcm 2&2&2\ecm$}\\
(b) Same approach on B:\\
 \centerline {$P_B = B(B^*B)^{-1}B^* = \frac{1}{6}\bcm 5&2&1\\2&2&-2\\1&-2&5\ecm$}\\
And the image under $P$ of the vector $(1,2,3)^*$ is:\\
\centerline {$\frac{1}{6}\bcm 5&2&1\\2&2&-2\\1&-2&5\ecm \bcm 1\\2\\3\ecm = \bcm 2&0&2\ecm$}\\

% insert your solution here!


%--------------------------------------------------------------------------
\vskip 1cm
\hrule
{\bf Problem 6.}
Exercise 7.1 in Trefethen and Bau.


% uncomment the next two lines if you want to insert solution...
%\vskip 1cm
%{\bf Solution:}

% insert your solution here!

%--------------------------------------------------------------------------
\vskip 1cm
\hrule
{\bf Problem 7.}
Exercise 7.2 in Trefethen and Bau.


% uncomment the next two lines if you want to insert solution...
%\vskip 1cm
%{\bf Solution:}

% insert your solution here!

%--------------------------------------------------------------------------
\vskip 1cm
\hrule
{\bf Problem 8.}
Exercise 8.2 in Trefethen and Bau.

You can write a Python function instead of Matlab if you prefer.

Test your routine on the matrices from Exercise 6.4 of Trefethen and Bau
and submit your output.

Test it on other matrices of different sizes to convince yourself it is
working properly. You do not need to turn in more output, but submit the
code in a file {\tt mgs.m} or {\tt mgs.py} in a way that we can test it on other
matrices of our choosing.

Please write readable code!


% uncomment the next two lines if you want to insert solution...
%\vskip 1cm
%{\bf Solution:}

% insert your solution here!

\end{document}

